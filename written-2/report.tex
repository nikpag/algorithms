\input{/home/nick/latex-preambles/xelatex.tex}

\newcommand{\imagesPath}{.}

\title{
	Αλγόριθμοι και Πολυπλοκότητα \\
	2η σειρά γραπτών ασκήσεων
}
\author{Νικόλαος Παγώνας, el18175}
\date{}

\begin{document}
	\renewcommand{\setminus}{\mathbin{\backslash}}
	\maketitle
	
	\section*{Άσκηση 1: Δίσκοι και Σημεία}
	
		Για κάθε σημείο $p$ βρίσκουμε τα δύο σημεία $p_1, p_2$ που βρίσκονται πάνω στην $l$ και απέχουν απόσταση $r$ από το $p$. Έτσι, το κριτήριο κάλυψης του $p$ από δίσκο είναι αν υπάρχει δίσκος με κέντρο ανάμεσα στα $p_1, p_2$. \\
		
		Κάθε ζεύγος $p_1, p_2$ ορίζει ένα διάστημα πάνω στην ευθεία $l$, οπότε έχουμε $n$ διαστήματα. \\ 
		
		Το πρόβλημα πλέον ανάγεται στο να υπολογίσουμε τον ελάχιστο αριθμό σημείων πάνω στην ευθεία $l$ ώστε κάθε διάστημα να περιέχει τουλάχιστον ένα σημείο. \\
		
		Ο αλγόριθμος έχει ως εξής: 
		
		\begin{itemize}
			\item Ταξινομούμε τα διαστήματα με βάση το τέλος τους. Έστω ότι μετά την ταξινόμηση έχουμε $s_1, ..., s_n$ τις αρχές των διαστημάτων και $f_1, ..., f_n$ τα τέλη τους. 
			\item Δημιουργούμε μία (αρχικά κενή) λίστα η οποία θα περιέχει τα ζητούμενα σημεία. Έστω $p_\text{top}$ το πιο πρόσφατο στοιχείο που έχουμε προσθέσει στη λίστα κάθε φορά.
			\item Για $i = 1, 2,...,n$:
				\begin{itemize}
					\item Αν $s_i \leq p_\text{top}$, τότε το $i$-οστό διάστημα περιέχει το $p_\text{top}$ (επειδή τα διαστήματα είναι ταξινομημένα με βάση το τέλος τους), οπότε δεν κάνουμε τίποτα. 
					\item Αν $s_i > p_\text{top}$, τότε βάζουμε το $f_i$ στην λίστα των σημείων. 
				\end{itemize}
		\end{itemize}  
	
		Η πολυπλοκότητα του αλγορίθμου είναι $O(n \log n)$ λόγω της ταξινόμησης. \\
		
		\textbf{Απόδειξη ορθότητας:}
		
		Έστω ότι τα σημεία που δίνει ο greedy αλγόριθμος είναι τα $g_1, ..., g_m$ και αυτά που δίνει ο optimal/best αλγόριθμος είναι τα $b_1, ..., b_k$. Και στις δύο περιπτώσεις τα σημεία είναι ταξινομημένα από αριστερά προς τα δεξιά. \\
		
		Αποδεικνύουμε (μέσω επαγωγής) ότι για κάθε $l \leq k$ ισχύει $g_l \geq b_l$. Η βάση της επαγωγής είναι $l = 1$. Και έχουμε $g_l \geq b_l$ για να καλύψουμε το πρώτο διάστημα. \\
		
		Επαγωγική υπόθεση: $g_{l-1} \geq b_{l-1}$ \\
		
		Τώρα θα δείξουμε ότι $g_l \geq b_l$. Έστω $a_i$ το αμέσως επόμενο διάστημα που έχει $s_i > g_{l-1}$. Ο greedy αλγόριθμος δημιουργεί ένα καινούργιο σημείο $g_l = f_i$ για να καλύψει το διάστημα αυτό. Όμως αν λάβουμε υπόψιν μας ότι $b_{l-1} \leq g_{l-1} \leq s_i$, αυτό σημαίνει ότι το $b_{l-1}$ δεν καλύπτει το $s_i$. Δηλαδή το $b_l$ οφείλει να καλύψει το $a_i$ $(s_i \leq b_l \leq f_i)$. Άρα τελικά $b_l \leq g_l$. \\
		
		Παρομοίως μπορούμε να δείξουμε και ότι $m = k$. Ας υποθέσουμε ότι $m > k$. Αν ο greedy αλγόριθμος χρειάζεται να βάλει κάποιο extra σημείο μετά το $g_k$, αυτό θα σημαίνει ότι υπάρχει ένα διάστημα (έστω $a_i$) που έχει $s_i > g_k$. Όμως αν λάβουμε υπόψιν μας ότι $b_k \leq g_k$, τότε $b_k < s_i$. Επομένως το $b_k$ δεν καλύπτει το $a_i$. Όμως το $b_k$ είναι το τελευταίο σημείο της optimal/best λύσης, άτοπο.
	
	\section*{Άσκηση 3: Τοποθέτηση Στεγάστρων (και Κυρτό Κάλυμμα)}
		\subsection*{(α)}
			Θα χρησιμοποιήσουμε Δυναμικό Προγραμματισμό. Η αναδρομική σχέση που περιγράφει το πρόβλημα είναι η εξής:
			
			\[
				c(i) = \min_{0 \leq j \leq i}\left\{c(j-1) + (x_i - x_j)^2 + C\right\}
			\]
			
			όπου $c(i)$ είναι το βέλτιστο κόστος για να στεγάσουμε τα σημεία από $x_0$ έως και $x_i$. Ορίζουμε $c(-1) = 0$ για να καλύψουμε την ειδική περίπτωση όπου έχουμε ένα ενιαίο στέγαστρο από το $x_0$ μέχρι και το $x_i$.
			
			Σε κάθε βήμα θα πρέπει να βρίσκουμε το ελάχιστο μεταξύ $i$ στοιχείων, επομένως η πολυπλοκότητα του αλγορίθμου θα είναι:
			
			\[
				1 + 2 + 3 + ... = O(n^2).
			\] 
		
		\subsection*{(β)}
			Στην ουσία αναζητούμε μία κατάλληλη διάταξη των ευθειών $[l_1, l_2,...,l_m]$ τέτοια ώστε η ευθεία $l_1$ να έχει ελάχιστη τιμή από το $-\infty$ μέχρι το σημείο τομής της με την $l_2$, η $l_2$ να έχει ελάχιστη τιμή από το σημείο τομής της με την $l_1$ έως το σημείο τομής της με την $l_3$ και ούτω καθεξής. Λόγω της ταξινόμησης των κλίσεων, κάθε νέα ευθεία θα μπαίνει στην διάταξη αυτή (και ενδεχομένως μπορεί να διαγράφει κάποιες από τις προηγούμενες ευθείες). Τελικά όμως κάθε ευθεία θα εισαχθεί και θα διαγραφεί το πολύ μία φορά, άρα η διάταξη αυτή φτιάχνεται σε $Θ(n)$. \\
			
			Αφού φτιάξουμε αυτή την διάταξη, παίρνουμε τα σημεία με τη σειρά, και για κάθε σημείο ξεκινάμε από την πρώτη ευθεία της διάταξης, βρίσκουμε ποια ευθεία καλύπτει το σημείο αυτό, και επιστρέφουμε την ευθεία αυτή. Λόγω της ταξινόμησης των σημείων όμως, είναι σίγουρο ότι η ευθεία που ελαχιστοποιεί το σημείο $x_{i+1}$ θα είναι πιο μετά από την ευθεία που ελαχιστοποιεί το $x_i$. \\
			
			Έτσι, θα χρειαστεί μία διάσχιση συνολικά, και έχουμε τελική πολυπλοκότητα $Θ(n+k)$. \\
			
			Για να εντάξουμε την παραπάνω λύση στο προηγούμενο πρόβλημα, πρέπει να αναδιατάξουμε την αναδρομική σχέση ως εξής: \\
			
			\begin{align*}
				c(i) &= \min_{0 \leq j \leq i} \left\{c(j - 1) + x_i^2 - 2x_ix_j + x_j^2 + C\right\} \\
				&= x_i^2 + C + \min_{0 \leq j \leq i} \left\{a_jx_i + b_j\right\}
			\end{align*}
			
			όπου $a_j = -2x_j$ και $b_j=c(j-1)+x_j^2$. Τώρα φαίνεται πως η αναδρομική σχέση μπορεί να λυθεί με άμεση εφαρμογή του παραπάνω αλγορίθμου σε γραμμικό χρόνο (εφόσον τόσο τα σημεία όσο και οι κλίσεις είναι ταξινομημένα).
	
	\section*{Άσκηση 5: Το Σύνολο των Συνδετικών Δέντρων}
		
		\subsection*{(α)}
			\begin{itemize}
				\item Έστω $T_1 \neq T_2$ συνδετικά δέντρα και έστω ακμή $e \in T_1 \setminus T_2 \neq \emptyset $. 
				\item Αυτό σημαίνει ότι η προσθήκη της $e$ στο $T_2$ θα δημιουργήσει κύκλο. Επίσης, αναγκαστικά θα υπάρχει τουλάχιστον μία ακμή $e'$ του κύκλου που δεν ανήκει στο $T_1$, γιατί αν όλες οι ακμές του κύκλου ανήκαν στο $T_1$, τότε το $T_1$ θα περιείχε κύκλο. 
				\item Αφαιρούμε την $e'$ από το $Τ_2 \cup \left\{e\right\}$ και προκύπτει πάλι συνεκτικό δέντρο, αφού χαλάσαμε τον κύκλο και επιπλέον έχουμε $n-1$ ακμές.
				\item Αυτό σημαίνει ότι υπάρχουν ακμές μεταξύ των δέντρων που είναι "ανταλλάξιμες" όσον αφορά την συνεκτικότητα.
				\item Το ζητούμενο προκύπτει ως το συμμετρικό του παραπάνω συμπεράσματος.  
			\end{itemize}
			
			
			Έστω τώρα $e' = (u, v)$. Ο αλγόριθμος που βρίσκει την ακμή $e'$ είναι ο εξής: 
			
			\begin{itemize}
				\item Με αρχή το $u$ κάνουμε DFS μέχρι να βρούμε μονοπάτι $M$ που να συνδέει τα $u, v$ στο $T_1$.
				\item Για κάθε ακμή $e$ του μονοπατιού $M$ ελέγχουμε αν ανήκει στο $T_2$ και όταν βρούμε $e \in M$ τέτοια ώστε $e \notin T_2$, την αφαιρούμε. 
			\end{itemize} 	
		
			Το DFS κοστίζει $O(|V|)$ και ο έλεγχος $e \in T_2$ κοστίζει $O(1)$, άρα συνολικά έχουμε κόστος $O(|V|)$.
			
		\subsection*{(β)}
			Έστω $T_1, T_2$ δύο συνεκτικά δέντρα που ανήκουν στο $H$ και $d(T_1, T_2)$ η απόσταση των $T_1, T_2$ στο $H$. Θα χρησιμοποιήσουμε επαγωγή για να δείξουμε ότι $|Τ_1 \setminus Τ_2| = k$ αν $d(T_1, T_2) = k$. \\
			
			\textbf{Βάση:} Εξ' ορισμού γνωρίζουμε ότι όταν $d(T_1, T_2) = 1$ τότε $|T_1 \setminus T_2| = 1$. Αυτή είναι και η βάση μας $(k = 1)$. \\ 
			
			\textbf{Επαγωγική υπόθεση:} Έστω ότι $|T_1 \setminus T_2| = k$ αν $d(T_1, T_2) = k$. \\
			
			\textbf{Επαγωγικό βήμα:} Θα δείξουμε το ζητούμενο για $k + 1$. 
			
			\begin{itemize}
				\item Αφού $d(T_1, T_2) = k + 1$ τότε θα υπάρχει $T' \in H$ τέτοιο ώστε $d(T_1, T') = 1$ και $d(T', T_2) = k$.
				\item Από την επαγωγική υπόθεση γνωρίζουμε ότι $|T' \setminus T_2| = k$ και από την βάση γνωρίζουμε ότι $|Τ_1 \setminus T'| = 1$. Άρα θα ισχύει είτε $|T_1 \setminus T_2| = k - 1$ είτε $|T_1 \setminus T_2| = k + 1$.
				\item Αν $|Τ_1 \setminus T_2| = k - 1$ τότε από επαγωγική υπόθεση προκύπτει ότι $d(T_1, T_2) = k - 1$ το οποίο είναι άτοπο αφού έχουμε υποθέσει ότι $d(T_1, T_2) = k + 1$.
				\item Επομένως, αναγκαστικά προκύπτει ότι $|T_1 \setminus T_2| = k + 1$.	
			\end{itemize}
		
		Ο αλγόριθμος για τον υπολογισμό ενός συντομότερου μονοπατιού είναι ο εξής:
		
		\begin{itemize}
			\item Υπολογίζουμε το σύνολο ακμών του $T_2 \setminus T_1$.
			\item Για κάθε ακμή αυτού του συνόλου προσθέτουμε την ακμή $e$ στο υπάρχον δέντρο.
		\end{itemize}    
			
		Ο αλγόριθμος είναι ορθός διότι αν $T_2 \setminus T_1 = k$ τότε χρειαζόμαστε $k$ βήματα για να μετατρέψουμε το $T_1$ στο $T_2$. Επομένως έχουμε βρει ένα μονοπάτι μήκους $k$ στο $H$, δηλαδή το συντομότερο μονοπάτι. \\
		
		Όσον αφορά το κόστος του αλγορίθμου, έχουμε τα εξής: 
		
		\begin{itemize}
			\item Σε $O(|V|)$ αποθηκεύουμε το $T_1$ ως ένα δέντρο όπου κάθε κόμβος δείχνει στον πατέρα του. 
			\item Για κάθε ακμή $e \in T_2$ ελέγχουμε σε σταθερό χρόνο αν $e \in T_1$
			\item Έτσι κάνουμε $k$ updates στις ακμές, και κάθε update χρειάζεται $O(|V|)$. 
		\end{itemize}
	
		Επομένως, η συνολική πολυπλοκότητα του αλγορίθμου είναι $O(k\cdot|V|)$.
		
		\subsection*{(γ)}
			Για να λύσουμε αυτό το πρόβλημα, αρχικά θα φτιάξουμε το MST (μία φορά) και για κάθε ακμή $e$ που δεν ανήκει σ' αυτό, θα βρίσκουμε το MST που την περιέχει, βάση του MST που φτιάξαμε αρχικά. Αυτό γίνεται αν προσθέσουμε την $e$ στο MST και μετά αφαιρέσουμε την βαρύτερη ακμή στον κύκλο που δημιουργείται. \\
			
			Υπολογίζουμε λοιπόν το MST (έστω $T$) σε $O(|E|log|E|)$ με Kruskal. Για την εύρεση του κύκλου που περιέχει την $e$, μπορούμε να κάνουμε ένα DFS (σε $O(|V|)$) από κάθε κορυφή $u$ και κρατάγαμε το βάρος της βαρύτερης ακμής στο μονοπάτι $u - v$ πάνω στο $Τ$, για κάθε άλλη κορυφή $v \neq u$. Συνολικά λοιπόν θα είχαμε χρόνο $O(|V|^2)$.
\end{document}